\section{Tipos, Clases y Módulos}


\begin{verbatim}
>> obj = [1, {:a => 2}]
=> [1, {:a=>2}]
>> obj.class                      # retorna la clase de un objeto
=> Array
>> obj.class.superclass           # retorna la superclase de un objeto
=> Object
>> obj.instance_of? Object        # determina si obj.class == Object
=> false
>> obj.instance_of? Array
=> true
>> obj.is_a? Object               # determina si obj es de una subclase de Object
=> true
>> obj.is_a? Array
=> true
>> obj.kind_of? Object            # kind_of? es un sinónimo de is_a?
=> true
>> Array === obj                  # equivalente a obj.is_a? Array
=> true
>> Object === obj
=> true
>> obj.respond_to? 'each'        # si tiene un método público o protected llamado 'each'
=> true                          # si se le pasa true como segundo argumento se comprueban
                                 # también los privados
>> Array.instance_methods(false) 
=> ["insert"
 "sort" "include?" "size" "&" "to_ary" "clear" "yaml_initialize" "shuffle"
 "replace" "pack" "zip" "flatten!" "to_s" "pop" "pretty_print_cycle" "hash"
 "cycle" "*" "indices" "nitems" "index" "collect" "+" "compact!"
 "last" "rassoc" "count" "drop" "delete" "delete_at" "combination" "collect!"
 "select" "each_index" "-" "flatten" "eql?" "fill" "length" "uniq!"
 "at" "choice" "reject!" "[]" "take" "inspect" "shift" "compact"
 "pretty_print" "[]=" "|" "find_index" "slice!" "each" "empty?" "transpose"
 "<<" "frozen?" "rindex" "map" "reverse_each" "reverse!" "to_a" "push"
 "uniq" "delete_if" "first" "product" "drop_while" "concat" "reject"
 "map!" "join" "slice" "indexes" "taguri" "<=>" "assoc" "fetch" "to_yaml"
 "==" "values_at" "permutation" "take_while" "unshift" "reverse"
 "sort!" "shuffle!"  "taguri="]
\end{verbatim}

  \subsection{Antepasados y Módulos}
  
Los siguientes métodos sirven para determinar que módulos han sido incluídos 
y cuales son los ancestros de una clase o módulo:
\begin{verbatim}
[~/chapter8ReflectionandMetaprogramming]$ cat ancestryAndModules.rb 
module A; end

module B; include A; end

class C; include B; end

if __FILE__ == $0
  puts C < B               # => true: C includes B
  puts B < A               # => true: B includes A
  puts C < A               # => true
  puts Fixnum < Integer    # => true: all fixnums are integers
  puts Integer <Comparable # => true: integers are comparable
  puts Integer < Fixnum    # => false: not all integers are fixnums
  puts (String < Numeric).inspect  # => nil: strings are not numbers
  
  puts A.ancestors.inspect         # => [A]
  puts B.ancestors.inspect         # => [B, A]
  puts C.ancestors.inspect         # => [C, B, A, Object, Kernel, BasicObject]
  puts String.ancestors.inspect    # => [String, Comparable, Object, Kernel, BasicObject]
 
  puts C.include?(B)       # => true
  puts C.include?(A)       # => true
  puts B.include?(A)       # => true
  puts A.include?(A)       # => false 
  puts A.include?(B)       # => false

  puts A.included_modules.inspect  # => []
  puts B.included_modules.inspect  # => [A]
  puts C.included_modules.inspect  # => [B, A, Kernel]
end
\end{verbatim}

